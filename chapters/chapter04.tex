\chapter{Advanced Concepts \& Usage}

\section{Mode \& Instantiatedness}

\subsection{Higher-order insts and modes}

\section{Multiple solutions \& Determinism}

As we have seen in Chapter 2, in Mercury, the following determinism categories exist:

\begin{itemize}
	\item Deterministic (\texttt{det}): guaranteed to have one and exactly one solution.
	\item Semideterministic (\texttt{semidet}): have exactly one solution, but does not guarantee to produce it.
	\item Multisolution (\texttt{multi}): guaranteed to have a solution among possibly many solutions.
	\item Nondeterministic (\texttt{nondet}): have possibly many solutions, does not guarantee to produce one.
	\item Failure (\texttt{failure}): cases where there's zero solutions. They are not actual errors but a part of the logic (e.g. arity mismatch during unification, which will never produce a solution because the arity is different).
	\item Errorneous (\texttt{errorneous}): also have zero solutions, but they \textbf{do} represent actual errors which in other languages would be represented in the form of runtime exception throw or panic.
\end{itemize}


\subsection{Commited-Choice Determinism}

\section{Definite Clause Grammar}

\section{Typeclasses}

\section{Existential Types}

\section{Purity System}

\section{Working with other people's libraries}


%%% Local Variables:
%%% mode: LaTeX
%%% TeX-master: "../main"
%%% End:
